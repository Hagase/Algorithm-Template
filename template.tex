\documentclass[a4paper,11pt]{article}
\usepackage{zh_CN-Adobefonts_external} % Simplified Chinese Support using external fonts (./fonts/zh_CN-Adobe/)
\usepackage{fancyhdr}  % 页眉页脚
\usepackage{minted}    % 代码高亮
\usepackage[colorlinks]{hyperref}  % 目录可跳转
\setlength{\headheight}{15pt}

% 定义页眉页脚
\pagestyle{fancy}
\fancyhf{}
\fancyhead[C]{Algorithm Library}
\lfoot{}
\cfoot{\thepage}
\rfoot{}

\author{}   
\title{Algorithm Library}

\begin{document} 
\maketitle % 封面
\newpage % 换页
\tableofcontents % 目录
\newpage

\section{头文件}
\inputminted[breaklines]{c++}{pre/pre.cpp}

\section{暴力}
\subsection{枚举}
\inputminted[breaklines]{c++}{violence/permutation.cpp}
\subsection{子集生成}
\inputminted[breaklines]{c++}{violence/subset.cpp}
\subsection{回溯}
\inputminted[breaklines]{c++}{violence/backtracking.cpp}


\section{搜索}
\subsection{三分法}
\inputminted[breaklines]{c++}{search/ternarySearch.cpp}
\subsection{dfs}
\inputminted[breaklines]{c++}{search/dfs.cpp}
\subsection{bfs}
\inputminted[breaklines]{c++}{search/bfs.cpp}


\section{数据结构}
\subsection{并查集}
\inputminted[breaklines]{c++}{ds/unionfind.h}
\subsection{heap}
\inputminted[breaklines]{c++}{ds/heap.cpp}
\subsection{二叉树}
\subsubsection{各种遍历及合成}
\inputminted[breaklines]{c++}{ds/bitree.cpp}

\section{动态规划}

\subsection{背包}
\subsubsection{01背包}	
\inputminted[breaklines]{c++}{dp/01bag.cpp}
\subsubsection{多重背包}
\inputminted[breaklines]{c++}{dp/multibag.cpp}
\subsection{背包-其他}
\inputminted[breaklines]{c++}{dp/bag_others.cpp}
\subsection{最长公共子序列}
\inputminted[breaklines]{c++}{dp/lcs.cpp}

\subsection{最长上升子序列}
\inputminted[breaklines]{c++}{dp/lis.cpp}
\subsection{最长公共上升}
\inputminted[breaklines]{c++}{dp/lcis.cpp}
\subsection{maxsubsum}
\inputminted[breaklines]{c++}{dp/max_sub_sum.cpp}
\subsection{maxsubmatsum}
\inputminted[breaklines]{c++}{dp/max_sub_mat_sum.cpp}
\subsection{tsp}
\inputminted[breaklines]{c++}{dp/tsp.cpp}
\subsection{拆分方案数}
\inputminted[breaklines]{c++}{dp/split_scheme.cpp}

\section{图论}
\subsection{tree}
\inputminted[breaklines]{c++}{graph/tree.h}
\subsection{拓扑排序}
\inputminted[breaklines]{c++}{graph/topoord.cpp}
\subsection{单源最短路径}
\subsubsection{Bellman-Ford}
\inputminted[breaklines]{c++}{graph/bellman.cpp}
\subsubsection{Dijkstra}
\inputminted[breaklines]{c++}{graph/dijkstra.cpp}

\subsection{多源最短路径}
\subsubsection{Floyd}
\inputminted[breaklines]{c++}{graph/floyd.cpp}

\subsection{最小生成树} 
\subsubsection{Prim} 
\inputminted[breaklines]{c++}{graph/prim.cpp}
\subsubsection{Kruskal} 
\inputminted[breaklines]{c++}{graph/kruskal.cpp}



\section{数论}
\subsection{进制转换}
\inputminted[breaklines]{c++}{number_theory/base_conversion.cpp}
\subsection{gcd}
\inputminted[breaklines]{c++}{number_theory/gcd.cpp}
\subsection{改进素数筛}
\inputminted[breaklines]{c++}{number_theory/prime.cpp}
\subsection{模线性方程}
\inputminted[breaklines]{c++}{number_theory/modequ.cpp}

\subsection{中国剩余定理}
\inputminted[breaklines]{c++}{number_theory/china.cpp}
\subsection{快速幂}
\inputminted[breaklines]{c++}{number_theory/pow_mod.cpp}





\section{字符串}
\subsection{字典树}
\inputminted[breaklines]{c++}{string/trie.cpp}
\subsection{KMP}
\inputminted[breaklines]{c++}{string/kmp.cpp}
\subsection{hash}
\inputminted[breaklines]{c++}{string/hash.cpp}
\subsection{表达式计算}
\subsubsection{前缀表达式}
\inputminted[breaklines]{c++}{string/pre_expr.cpp}

\subsubsection{中缀表达式}
\inputminted[breaklines]{c++}{string/post_expr.cpp}
\subsection{find longest shortest word}
\inputminted[breaklines]{c++}{string/find_longest_shortest_word.cpp}

\section{bigint}
\inputminted[breaklines]{c++}{bigint/bigint.cpp}

\end{document}
